\chapter*{Abstract}\label{ch:abstract}
% 350 WORDS MAXIMUM
Modular design has been the cornerstone of contemporary engineering, enabling efficient production of exchangeable parts that interact in a reproducible manner to constitute functional systems.
    In this proposed thesis, we transfer engineering modular design principles to the emerging fields of synthetic biology and metabolic engineering, that have promising applications to address problems related to health, energy, security, and the environment. In particular, we focus on microbial biocatalysis which has the potential to become a renewable and lower-cost replacement of traditional chemical synthesis processes.
    The thesis begins with an interdisciplinary review and pespective of the concepts, methodology, and applications of modular design.
    Then, a conceptual, mathematical, and algorithmic framework based on multi-objective optimization is developed to design modular cell biocatalysts.
    Genome-scale metabolic models that account for all known metabolic reactions of industrially-relevant organisms such as \textit{Escherichia coli} and \textit{Clostridium thermocellum} are used as a basis for simulation of biocatalytic strain phenotypes.
    In combination with these models, the proposed framework is used to design modular cell systems for renewable production of diverse biofuels and biochemicals.% in major industrial organisms such as \textit{Escherichia coli} and \textit{Clostridium thermocellum}.  % TODO: No modular cell designs for C. therm are presented, so how to rephrase this or put c. therm chapter in context??
    %The basis of this analysis are genome-scale metaboli models
    %In addition to this analysis, genome-scale metabolic models and related techniques are applied and developed are app
    Overall this contribution addresses the current interest in modular design in synthetic biology through novel systematic principles and quantitative tools.
    We anticipate this modular cell design approach not only brings whole-cell biocatalysis closer to being an industrially competitive technology, but also provides tools to understand the natural modular architectures of metabolic networks designed by evolution for billions of years under physical and biological constraints.
%The modular cell design approach aims to overcome the prohibitively slow and costly research and development cycles in strain biocatalysis by eliminating redundancy and increasing robustness, pushing whole-cell biocatalysis into an industrially competitive technology.
