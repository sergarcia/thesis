%%%%%%%%%%%%%%%%%%%%%%%%%%%%%%%%
%%%% 6-steps/sentences method %%
%%%%%%%%%%%%%%%%%%%%%%%%%%%%%%%%
%  1. Introduction.  what’s the topic?
%  2. State the problem you tackle. What’s the key research question?
%  3. Summarize why nobody else has adequately answered the research question yet.
%  4. Explain how you tackled the research question. What’s your big new idea?
%  5. How did you go about doing the research that follows from your big idea. Did you run experiments? Build a piece of software? Carry out case studies?
%  6. What’s the key impact of your research? What’s it all mean? Why should other people care? What can they do with your research.
%%%%%%%%%%%%%%%%%%%%%%%%%%%%%%%%
\chapter*{Abstract}\label{ch:abstract}
% 350 WORDS MAXIMUM
Modular design has been the cornerstone of contemporary engineering, enabling efficient production of exchangeable parts that interact in a reproducible manner to constitute functional systems.
    In this thesis, we transfer engineering modular design principles to the emerging fields of synthetic biology and metabolic engineering, that have promising applications to address problems related to health, energy, security, and the environment.
    We focus on microbial biocatalysis which can become a renewable and lower-cost replacement of traditional chemical synthesis processes.
    This thesis begins with an interdisciplinary review and perspective of the concepts, methodology, and applications of modular design.
    Then, we develop a conceptual, mathematical, and algorithmic framework based on multi-objective optimization theory to design modular cell biocatalysts.
    The proposed framework is used to design modular cell systems for renewable production of diverse biofuels and biochemicals, using
    genome-scale metabolic models of the organisms \textit{Escherichia coli} and \textit{Clostridium thermocellum} to simulate metabolic phenotypes.
    Overall, this contribution addresses the current interest in modular design in synthetic biology through novel systematic principles and quantitative tools.
    We anticipate this modular cell design approach will not only bring whole-cell biocatalysis closer to being an industrially competitive technology, but also provide tools to understand the natural modular architectures of metabolic networks designed by evolution for billions of years under biological constraints.
%The modular cell design approach aims to overcome the prohibitively slow and costly research and development cycles in strain biocatalysis by eliminating redundancy and increasing robustness, pushing whole-cell biocatalysis into an industrially competitive technology.
