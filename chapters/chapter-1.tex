\chapter{Introduction} \label{ch:introduction}

% SG Notes:
% - Highlight hypothesis and how does it improve over existing knowledge
% - When refering to each chapter note if their correspondence to publications. (or have a sentence "the chapters of these thesis correspond to publications in peer-review journals\cite{}")
% - When refering to the chapters, note co-author publications not discussed here.
% -
% (Revisit intro to thesis proposal)
% - It is nice how statistical mechanics bridges the gap between phenomenological models and microscopic principles (at least in thermodynamics). Are the interactions of biological molecules too heterogenous to be explained in these terms?
% - Cite all my papers (both first author and coauthor)

% Paragraph structure:
% 1. Developments in biology and biotechnology and the challenge ahead
% 2. Hypothesis and comparison to existing modular design views
% 3. Developments/goals of the thesis



% TODO: This view needs to be refined. Why bring up the Demon? What is the gap? The gap is between first principles and emergent behavoir
% Note: In terms of biochemistry we might have sufficient knowledge, but regarding biophysicis there might be novel phenomena (e.g., catch bonds) that we still have not thought of and has great implications to system behavior.
% The point is rather than we lack information and computational power, rather than first principles. Hence we need models that bridge the gap between first principles and emergent phenomena, without requiring the omniscence Laplace's demon.
% Laplaces demon might not fit well here after all

% 1. Developments in biology and biotechnology and the challenge ahead
Vitalism was the notion that life arises from an essential principle that cannot be explained in terms of physical and chemical phenomena.
Eduard Buchner's discovery of what would later be called enzymes during the late 19th century provided the final blow to vitalism and opened the door to
great developments in our mechanistic understanding of living systems throught the 20th century.
%After the final blow to this theory provided by Eduard Buchner's discovery of what would later be called enzymes during the late 19th century, the mechanistic understanding of living systems greatly developed over the 20th century.
However, it was not until the last few decades that bioengineers began to manipulate cellular DNA for diverse applications ranging from renewable chemical production to medical treatment. Despite recent success stories, such applied fields remain highly constrained by a lack of models with sufficent predictive and explanatory power to bridge the gap between first principles and emerging phenomena.
%Our current understanding of fundamental biological phenomena is sufficient to make a biological version of Laplace's Demon theoretically plausible, .i.e., if we were to know all biochemical and biophysical events of an organism and their parameters we could predict all biological behavior given sufficient computational power. However, other areas of science need not to relay on such Demon for their application, but rather phenomenological and statistical models with high predictive power.
%(models with sufficient explanatory power to conduct chemical transformations and treat diseases)
Finding such models for biological systems might be an unprecedented challenge, and remains a remarkable opportunity for scientific discovery.
% Bridge the gap between first principles and emergent behavoir
In addition to this scientific challenge, our increasing capabilities to engineer biological system need to be complemented by principles and design methods that will ensure efficient development of predictable systems.
Among potential applications, biocatalysis technologies can broaden the accessibility of goods, including food and medication, and reduce global warming, two issues that will likely become the major drivers of civil conflict during the 21st century. % Last sentence is not very clear


%  TODO: this paragraph needs to transition to the focus on biocatalysis !!
% 2. Hypothesis and comparison to existing modular design views
The most important concept is modern engineering is modular design. Computers, vehicles, factories, and many other complex devices are often assembled from exchangeable units of self-contained functionality known as modules.
By analogy with electrical engineering, synthetic biologist have aimed to build biological parts that enable modular assembly and re-usability in different contexts.
However, most modularization efforts in this field have been applied conceptually and are often applied in isolation to specific cellular components (e.g., a metabolic pathway) rather than seeing the cell as a chassis and components and modules. Hence they are difficult to generalize and do not take advantage of the potential to build platform strains.
The primary aim of this thesis is to advance modular design principles in synthetic biology and metabolic engineering with emphasis on biocatalysis applications.
Unlike previous modularization approaches, this work considers modular design at the system level and provides quantitative tools for its application.
Chapter~\ref{ch:review} reviews the engineering and bioengineering literature to develop this view in depth.



%Conceptually, modular design is often discussed in synthetic biology, in particular since analogies with electronical engineering are common.
%This modularization approach is unlike exiting methods in synthetic biology that focus on the ... the primary differnece is that we focus on modularization at the system level,  that is we consider how to design modules, interfaces, and a chassis in an integrated fashion.

%We propose a framework based on systems-level mechanistic biological modelsand optimization principles

%%-> Introduce challenge to current modularization in biology
%%-> Refer to Chapter 2 for in-depth investigation of these topics
%%-> Proposed approach and how it differs from existing methods

%\textbf{The goal of this thesis is to advance modular design principles in synthetic biology and metabolic engineering with emphasis on biocatalysis applications.}
%%\quotecolor{ % paper 3
%To date, modular design in metabolic engineering has been mostly applied at the pathway level, where enzyme module expression is adjusted to increase target metabolite production.\citep{biggs2014, jeschek2017, lu2018, yadav2012} More recently, a system-level modular cell design (ModCell) has been proposed.\citep{trinh2015}
%Unlike conventional modular pathway optimization methods that target one product, ModCell enables rapid construction of multiple production strains each synthesizing a different product.
%Each optimal production strain is obtained by assembling a reusable modular (chassis) cell with an exchangeable production module(s) in a plug-and-play fashion, resembling the advantages of modular design in traditional engineering disciplines.
%Specifically, a modular cell contains core metabolic phenotypes shared among production modules (Figure~\ref{fig:modcell_concept}.A).
%The chassis interfaces with the modules through enzymatic and genetic synthesis machinery and precursor metabolites (Figure~\ref{fig:modcell_concept}.B).
%Modules contain auxiliary regulatory and metabolic pathways (Figure~\ref{fig:modcell_concept}.C) that enable a desired phenotype for optimal biosynthesis of a target molecule, such as growth-coupled-to-product formation (\textit{GCP} design) or stationary-phase product synthesis (\textit{NGP} design) (Figure~\ref{fig:modcell_concept}.D).
%%} % paper 3

% TODO: Just review, but this should be mostly the same
% 3. Developments/goals of the thesis
This thesis aims to attain three goals:
i) The development of general modular cell design principles and associated mathematical models (Chapters~\ref{ch:review}, \ref{ch:modcell2});
ii) the development of scalable algorithms to solve the mathematical models (Chapters~\ref{ch:moea}, \ref{ch:milp}, \ref{ch:hpc});
and iii) the application of the resulting modular cell design tool to understand driving principles of natural biological modularity and to design modular biocatalyst strains based on industrially-relevant hosts and production modules (Chapters~\ref{ch:modcell2}, \ref{ch:milp}, \ref{ch:ctherm}, \ref{ch:hpc}).
In conclusion, the outcome of this research is to bring modularity principles proven in conventional engineering to metabolic engineering, leading to increased robustness and efficiency thereby accelerating the strain design process that remains the major roadblock for widespread industrial application of microbial catalysis.

