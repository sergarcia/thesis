
\renewcommand{\hbAppendixPrefix}{C}

\renewcommand{\thefigure}{\hbAppendixPrefix\arabic{figure}}
\setcounter{figure}{0}
\renewcommand{\thetable}{\hbAppendixPrefix\arabic{table}}
\setcounter{table}{0}
\renewcommand{\theequation}{\hbAppendixPrefix\arabic{equation}}
\setcounter{equation}{0}

\section{Supplementary Material 1 for Chapter \ref{ch:ctherm}} \label{apx:sm1-ctherm}

%\documentclass[dvipsnames]{article}
%\usepackage[sm]{paper}
%%\usepackage{longtable}
%\usepackage{supertabular}
%\addbibresource{bibliography.bib}

% Command to easily format gene knockouts
%\makeatletter
%\newcommand\ko[1]{{\@for\tmp:=#1\do{\textit{$\Delta$\tmp}}}}



%\title{ \textbf{Supplementary Material 1:} \\Development of an updated genome-scale metabolic model of \textit{Clostridium thermocellum} and its application for integration of multi-omics datasets}
%\author[1,2]{Sergio Garcia}
%\author[2,3]{R. Adam Thompson}
%\author[2,5]{Richard J. Giannone}
%\author[2,4]{Satyakam Dash}
%\author[2,4]{Costas D. Maranas}
%\author[1,2,3,*]{Cong T. Trinh}
%\affil[1]{Department of Chemical and Biomolecular Engineering, The University of Tennessee, Knoxville, TN, United States}
%\affil[2]{Center for Bioenergy Innovation, Oak Ridge National Laboratory Oak Ridge, TN, United States}
%\affil[3]{Bredesen Center for Interdisciplinary Research and Graduate Education, The University of Tennessee, Knoxville and Oak Ridge National Laboratory, Oak Ridge, TN, USA}
%\affil[4]{Department of Chemical Engineering, The Pennsylvania State University, University Park, PA, United States}
%\affil[5]{Chemical Sciences Division, Oak Ridge National Laboratory. Oak Ridge, TN, United States.}
%\affil[*]{Corresponding author: 1512 Middle Dr, DO432, Deparment of Chemical and Biomolecular Engineering, University of Tennesse, Knoxville, TN 37996, United States. Tel: 865-974-2181. Email: ctrinh@utk.edu.}
%
%
%\begin{document}
%\maketitle
%%\tableofcontents
%%\listoftables
%\newpage

%\section*{Consistent cases}

%\todo[inline]{A list of reaction abreviations to names can be added at the end but otherwise it is too much for one table, also consider adding the model in "table form" for easy reference}

%\todo[inline]{Add continuation caption}
% A way to add continuation caption:
%https://tex.stackexchange.com/questions/11380/how-to-repeat-top-rows-column-headings-on-every-page
% --- See longtable instead
% https://tex.stackexchange.com/questions/26462/make-a-table-span-multiple-pages


%\begin{table}[!ht]
%\caption[Consistent reactions between flux simulations and proteomic data]{The 70 consistent reactions in the
%\textit{$\Delta$hydG-$\Delta$ech} case study and their associated fold changes. The biomass reaction is not included due to size. This table is continued in the following pages.}
\small
\rowcolors{2}{gray!25}{white}
\begin{longtable}{lp{0.5\textwidth}ccc}
    \caption[Consistent reactions in the $\Delta$\textit{hydG}-$\Delta$\textit{ech} case study]{The 70 consistent reactions in the $\Delta$\textit{hydG}-$\Delta$\textit{ech} case study and their associated fold changes. The biomass reaction is not included due to size. This table is continued in the following pages.\label{foo}}\\\toprule
\rowcolor{white} \multirow{1}{*}{ID} & \multirow{1}{*}{Formula}  & \multicolumn{3}{c}{Fold change} \\
\rowcolor{white} & & \emph{proteomics} & \emph{pFBA} & \emph{FVAcenter} \\
\midrule
\endfirsthead
\caption* {\textbf{Table \ref{foo} Continued}}\\\toprule
\rowcolor{white} \multirow{1}{*}{ID} & \multirow{1}{*}{Formula}  & \multicolumn{3}{c}{Fold change} \\
\rowcolor{white} & & \emph{proteomics} & \emph{pFBA} & \emph{FVAcenter} \\
\midrule
\endhead % all the lines above this will be repeated on every page
%\hline
%\multicolumn{5}{r@{}}{continued \ldots}\\
\endfoot
\hline
\endlastfoot
%
%\tablehead{%
%\toprule
%\rowcolor{white} \multirow{1}{*}{ID} & \multirow{1}{*}{Formula}  & \multicolumn{3}{c}{Fold change} \\
%%\rowcolor{white} & & \emph{prote-\\omics} & \emph{pFBA} & \emph{FVA\\center} \\
%\rowcolor{white} & & \emph{proteomics} & \emph{pFBA} & \emph{FVAcenter} \\
%\midrule}
%\tabletail{\hline}
%\rowcolors{2}{gray!25}{white}
%\begin{supertabular}{lp{0.5\textwidth}ccc}
MDH	&	mal\_\_L\_c + nad\_c $\leftrightarrow$ h\_c + nadh\_c + oaa\_c	&	1.4	&	0.2	&	0.1	\\
PEPCK\_re	&	co2\_c + gdp\_c + pep\_c $\rightarrow$ gtp\_c + oaa\_c	&	0.9	&	0.1	&	0.1	\\
VOR2b	&	3mob\_c + coa\_c + 2.0 fdxo\_42\_c $\rightarrow$ co2\_c + 2.0 fdxr\_42\_c + h\_c + ibcoa\_c	&	0.9	&	12.3	&	0.8	\\
PFL	&	coa\_c + pyr\_c $\rightarrow$ accoa\_c + for\_c	&	0.5	&	0.2	&	0.8	\\
FRNDPR2r	&	2.0 fdxr\_42\_c + h\_c + nadh\_c + 2.0 nadp\_c $\leftrightarrow$ 2.0 fdxo\_42\_c + nad\_c + 2.0 nadph\_c	&	0.3	&	0.2	&	2.3	\\
ALCD2x	&	acald\_c + h\_c + nadh\_c $\rightarrow$ etoh\_c + nad\_c	&	0.1	&	0.8	&	0.6	\\
IBUTCOARx	&	h\_c + ibcoa\_c + nadh\_c $\rightarrow$ 2mppal\_c + coa\_c + nad\_c	&	0.1	&	12.3	&	0.8	\\
ACALD	&	accoa\_c + h\_c + nadh\_c $\rightarrow$ acald\_c + coa\_c + nad\_c	&	0.1	&	0.8	&	1.1	\\
ALCD23xi	&	2mppal\_c + h\_c + nadh\_c $\rightarrow$ ibutoh\_c + nad\_c	&	0.1	&	12.3	&	0.8	\\
ME2	&	mal\_\_L\_c + nadp\_c $\rightarrow$ co2\_c + nadph\_c + pyr\_c	&	0.1	&	0.3	&	0.1	\\
PSCVT	&	pep\_c + skm5p\_c $\rightarrow$ 3psme\_c + pi\_c	&	-0.1	&	-0.3	&	0.7	\\
PTAr	&	accoa\_c + pi\_c $\leftrightarrow$ actp\_c + coa\_c	&	-0.1	&	-2.7	&	-0.1	\\
    HSOR	&	3.0 h\_c + 3.0 nadph\_c + so3\_c $\rightarrow$ 3.0 h2o\_c + h2s\_c + 3.0 nadp\_c	&	-0.1	&	-0.3	&	-1.2	\\
TRDR	&	h\_c + nadph\_c + trdox\_c $\rightarrow$ nadp\_c + trdrd\_c	&	-0.1	&	-0.3	&	0.8	\\
IGPS	&	2cpr5p\_c + h\_c $\rightarrow$ 3ig3p\_c + co2\_c + h2o\_c	&	-0.1	&	-0.3	&	0.2	\\
GLUDy	&	glu\_\_L\_c + h2o\_c + nadp\_c $\leftrightarrow$ akg\_c + h\_c + nadph\_c + nh4\_c	&	-0.1	&	-0.5	&	-0.2	\\
ECH	&	2.0 fdxr\_42\_c + 3.0 h\_c $\leftrightarrow$ 2.0 fdxo\_42\_c + h2\_c + h\_e	&	-0.1	&	-15.6	&	-13.5	\\
ALLAS	&	24.0 ala\_\_D\_c + 24.0 atp\_c + 24.0 cdpglyc\_c + dg12dg\_c + 24.0 h2o\_c $\rightarrow$ ala\_lta\_c + 24.0 amp\_c + 24.0 cmp\_c + 48.0 h\_c + 24.0 ppi\_c	&	-0.1	&	-0.2	&	0.0	\\
HSDxi	&	aspsa\_c + h\_c + nadh\_c $\rightarrow$ hom\_\_L\_c + nad\_c	&	-0.1	&	-0.3	&	0.8	\\
PRMICI	&	prfp\_c $\rightarrow$ prlp\_c	&	-0.1	&	-0.3	&	-0.1	\\
OCBT	&	cbp\_c + orn\_\_L\_c $\leftrightarrow$ citr\_\_L\_c + h\_c + pi\_c	&	-0.2	&	-0.3	&	-0.3	\\
UAGCVT	&	pep\_c + uacgam\_c $\rightarrow$ pi\_c + uaccg\_c	&	-0.2	&	-0.3	&	-0.1	\\
ANS	&	chor\_c + gln\_\_L\_c $\rightarrow$ anth\_c + glu\_\_L\_c + h\_c + pyr\_c	&	-0.2	&	-0.3	&	0.2	\\
DHAD2	&	23dhmp\_c $\rightarrow$ 3mop\_c + h2o\_c	&	-0.2	&	-0.3	&	0.6	\\
ASPO2y	&	asp\_\_L\_c + nadp\_c $\rightarrow$ h\_c + iasp\_c + nadph\_c	&	-0.3	&	-0.3	&	-0.1	\\
NADK	&	atp\_c + nad\_c $\rightarrow$ adp\_c + h\_c + nadp\_c	&	-0.3	&	-0.3	&	-0.1	\\
METS	&	5mthf\_c + hcys\_\_L\_c $\rightarrow$ met\_\_L\_c + thf\_c	&	-0.3	&	-0.3	&	0.8	\\
IMPD	&	h2o\_c + imp\_c + nad\_c $\leftrightarrow$ h\_c + nadh\_c + xmp\_c	&	-0.3	&	-0.3	&	-0.3	\\
MG2abc	&	atp\_c + h2o\_c + mg2\_e $\rightarrow$ adp\_c + h\_c + mg2\_c + pi\_c	&	-0.3	&	-0.3	&	-0.1	\\
PDHam1hi	&	h\_c + pyr\_c + thmpp\_c $\rightarrow$ 2ahethmpp\_c + co2\_c	&	-0.4	&	-0.3	&	12.4	\\
ACAS\_2ahbut	&	2ahethmpp\_c + 2obut\_c $\rightarrow$ 2ahbut\_c + thmpp\_c	&	-0.4	&	-0.3	&	0.6	\\
%BIOMASS\_CELLOBIOSE	&	0.00749443085801535 3g12dgr\_SA2\_c + 0.000534651500429844 aglu\_lta\_c + 0.226981629081658 ala\_\_L\_c + 0.00159065471272163 ala\_lta\_c + 0.0116446298806306 amp\_c + 0.152834432829489 arg\_\_L\_c + 0.19337686956005 asn\_\_L\_c + 0.203565776332897 asp\_\_L\_c + 58.55453625355648 atp\_c + 0.00431378642200049 ca2\_c + 0.000648364692685445 cdp\_c + 0.000398993657037197 clpn\_SC\_c + 0.00255887932046522 cmp\_c + 0.089081651333124 cpd11452\_c + 0.00190186976521064 cpd12556\_c + 0.0240139232409941 ctp\_c + 0.0410739907802319 cys\_\_L\_c + 0.0151967584110132 datp\_c + 0.00837186690906119 dctp\_c + 0.00966229639458412 dg12dg\_c + 0.00837186690906119 dgtp\_c + 0.0151967584110132 dttp\_c + 0.00463796876834322 fe3\_c + 0.000492757166440938 gdp\_c + 0.0907475198501415 gln\_\_L\_c + 0.267038054788532 glu\_\_L\_c + 0.000598490485555795 glu\_lta\_c + 0.238689992949084 gly\_c + 0.00315836729016361 gly\_tea\_c + 0.00364879699360517 glygly\_tea\_c + 0.00131834154179374 gmp\_c + 0.0236155245744981 gtp\_c + 58.55453625355648 h2o\_c + 0.0505986081135555 his\_\_L\_c + 0.297230937300123 ile\_\_L\_c + 0.950718761041091 k\_c + 0.30404751643422 leu\_\_L\_c + 0.00128475957565977 lta\_c + 0.283396269735403 lys\_\_L\_c + 0.088041844113358 met\_\_L\_c + 0.136964880114158 mg2\_c + 0.00272678915113504 min\_tea\_c + 0.0403931203543032 nad\_c + 0.023341128936676 nadp\_c + 0.000523014185432926 nadph\_c + 0.0154610042101914 pg\_SC\_c + 0.149324601126697 phe\_\_L\_c + 0.123899561568787 pro\_\_L\_c + 0.0491959179126864 psetha\_BS\_c + 0.214473177933542 ser\_\_L\_c + 0.00580535770989122 tg12dg\_c + 0.181493989716975 thr\_\_L\_c + 0.0331727351396739 trp\_\_L\_c + 0.149596153059727 tyr\_\_L\_c + 0.0354242930957506 utp\_c + 0.251074983559902 val\_\_L\_c $\rightarrow$ 58.55453625355648 adp\_c + 58.55453625355648 h\_c + 58.55453625355648 pi\_c + 0.161830917308754 ppi\_c	&	-0.4	&	-0.3	&	-0.1	\\
GF6PTA	&	f6p\_B\_c + gln\_\_L\_c $\rightarrow$ gam6p\_c + glu\_\_L\_c	&	-0.4	&	-0.3	&	0.0	\\
CHRS	&	3psme\_c $\rightarrow$ chor\_c + pi\_c	&	-0.4	&	-0.3	&	0.7	\\
SERH	&	3ig3p\_c + ser\_\_L\_c $\rightarrow$ g3p\_c + h2o\_c + trp\_\_L\_c	&	-0.4	&	-0.3	&	0.2	\\
ALATA\_L	&	akg\_c + ala\_\_L\_c $\leftrightarrow$ glu\_\_L\_c + pyr\_c	&	-0.5	&	-0.3	&	-12.9	\\
NNDPR	&	h\_c + prpp\_c + quln\_c $\rightarrow$ co2\_c + nicrnt\_c + ppi\_c	&	-0.5	&	-0.3	&	-0.1	\\
TMDS	&	dump\_c + mlthf\_c $\rightarrow$ dhf\_c + dtmp\_c	&	-0.5	&	-0.3	&	-0.1	\\
PHETA1	&	akg\_c + phe\_\_L\_c $\leftrightarrow$ glu\_\_L\_c + phpyr\_c	&	-0.6	&	-0.3	&	0.8	\\
TYRTA	&	akg\_c + tyr\_\_L\_c $\leftrightarrow$ 34hpp\_c + glu\_\_L\_c	&	-0.6	&	-0.3	&	0.6	\\
GMPS	&	atp\_c + nh4\_c + xmp\_c $\rightarrow$ amp\_c + gmp\_c + 3.0 h\_c + ppi\_c	&	-0.6	&	-0.3	&	-0.3	\\
SKK	&	atp\_c + skm\_c $\rightarrow$ adp\_c + h\_c + skm5p\_c	&	-0.6	&	-0.3	&	0.7	\\
ACOTA	&	acorn\_c + akg\_c $\leftrightarrow$ acg5sa\_c + glu\_\_L\_c	&	-0.6	&	-0.3	&	-0.3	\\
PPDK	&	amp\_c + 2.0 h\_c + pep\_c + ppi\_c $\rightarrow$ atp\_c + pi\_c + pyr\_c	&	-0.7	&	-10.0	&	0.1	\\
GLUPRT	&	gln\_\_L\_c + h2o\_c + prpp\_c $\rightarrow$ glu\_\_L\_c + h\_c + ppi\_c + pram\_c	&	-0.7	&	-0.3	&	-0.3	\\
KARI	&	2ahbut\_c $\leftrightarrow$ cpd10162\_c	&	-0.8	&	-0.3	&	0.6	\\
KARI\_23dhmp	&	23dhmp\_c + nadp\_c $\leftrightarrow$ cpd10162\_c + h\_c + nadph\_c	&	-0.8	&	-0.3	&	0.6	\\
ARGSL	&	argsuc\_c $\leftrightarrow$ arg\_\_L\_c + fum\_c	&	-0.8	&	-0.3	&	-0.3	\\
LEUTA	&	4mop\_c + glu\_\_L\_c $\rightarrow$ akg\_c + leu\_\_L\_c	&	-0.8	&	-0.3	&	0.4	\\
ILETA	&	akg\_c + ile\_\_L\_c $\leftrightarrow$ 3mop\_c + glu\_\_L\_c	&	-0.8	&	-0.3	&	0.6	\\
VALTA	&	akg\_c + val\_\_L\_c $\leftrightarrow$ 3mob\_c + glu\_\_L\_c	&	-0.8	&	-1.4	&	-1.5	\\
ARGSS	&	asp\_\_L\_c + atp\_c + citr\_\_L\_c $\leftrightarrow$ amp\_c + argsuc\_c + 2.0 h\_c + ppi\_c	&	-0.8	&	-0.3	&	-0.3	\\
SHSL2	&	h2s\_c + suchms\_c $\rightarrow$ hcys\_\_L\_c + succ\_c	&	-0.9	&	-6.0	&	0.8	\\
AHSL	&	achms\_c + cys\_\_L\_c $\leftrightarrow$ ac\_c + cyst\_L\_c + h\_c	&	-0.9	&	-10.6	&	-0.3	\\
SHSL1	&	cyst\_L\_c + h\_c + succ\_c $\leftrightarrow$ cys\_\_L\_c + suchms\_c	&	-0.9	&	-10.6	&	0.4	\\
ACKr	&	actp\_c + adp\_c $\rightarrow$ ac\_c + atp\_c	&	-0.9	&	-2.7	&	-0.1	\\
QULNS	&	dhap\_c + iasp\_c $\rightarrow$ 2.0 h2o\_c + h\_c + pi\_c + quln\_c	&	-0.9	&	-0.3	&	-0.1	\\
NADS2	&	atp\_c + dnad\_c + gln\_\_L\_c + h2o\_c $\rightarrow$ amp\_c + glu\_\_L\_c + 2.0 h\_c + nad\_c + ppi\_c	&	-0.9	&	-0.3	&	-0.1	\\
FE3abc	&	atp\_c + fe3\_e + h2o\_c $\rightarrow$ adp\_c + fe3\_c + h\_c + pi\_c	&	-1.0	&	-0.3	&	-0.1	\\
ASPTA	&	akg\_c + asp\_\_L\_c $\leftrightarrow$ glu\_\_L\_c + oaa\_c	&	-1.0	&	-0.3	&	0.2	\\
CTPS1	&	atp\_c + nh4\_c + utp\_c $\rightarrow$ adp\_c + ctp\_c + 2.0 h\_c + pi\_c	&	-1.2	&	-0.3	&	0.0	\\
ACGK	&	acglu\_c + atp\_c $\rightarrow$ acg5p\_c + adp\_c	&	-1.2	&	-0.3	&	-0.3	\\
IGPDH	&	eig3p\_c $\rightarrow$ h2o\_c + imacp\_c	&	-1.2	&	-0.3	&	-0.1	\\
AGPR	&	acg5sa\_c + nadp\_c + pi\_c $\leftrightarrow$ acg5p\_c + h\_c + nadph\_c	&	-1.3	&	-0.3	&	-0.3	\\
PHEt2r	&	h\_e + phe\_\_L\_e $\leftrightarrow$ h\_c + phe\_\_L\_c	&	-1.5	&	-0.3	&	0.8	\\
UAG4Ei	&	uacgam\_c $\rightarrow$ udpacgal\_c	&	-1.5	&	-0.3	&	-0.1	\\
CYSS	&	acser\_c + h2s\_c $\rightarrow$ ac\_c + cys\_\_L\_c	&	-1.8	&	-0.3	&	-0.5	\\
BIF	&	2.0 fdxr\_42\_c + 3.0 h\_c + nadh\_c $\leftrightarrow$ 2.0 fdxo\_42\_c + 2.0 h2\_c + nad\_c	&	-1.8	&	-13.8	&	-12.5	\\
UMPK	&	atp\_c + h\_c + ump\_c $\rightarrow$ adp\_c + udp\_c	&	-2.1	&	-0.3	&	0.0	\\
SULabc	&	atp\_c + h2o\_c + so4\_e $\rightarrow$ adp\_c + h\_c + pi\_c + so4\_c	&	-4.6	&	-0.3	&	0.8	\\
\end{longtable}
\normalsize


%\end{table}

% Supertabular
%\topcaption[Consistent reactions between flux simulations and proteomic data]{The 70 consistent reactions in the
%\textit{$\Delta$hydG-$\Delta$ech} case study and their associated fold changes. The biomass reaction is not included due to size. This table is continued in the following pages.}
%\small
\rowcolors{2}{gray!25}{white}
\begin{longtable}{lp{0.5\textwidth}ccc}
    \caption[Consistent reactions in the $\Delta$\textit{hydG}-$\Delta$\textit{ech} case study]{The 70 consistent reactions in the $\Delta$\textit{hydG}-$\Delta$\textit{ech} case study and their associated fold changes. The biomass reaction is not included due to size. This table is continued in the following pages.\label{foo}}\\\toprule
\rowcolor{white} \multirow{1}{*}{ID} & \multirow{1}{*}{Formula}  & \multicolumn{3}{c}{Fold change} \\
\rowcolor{white} & & \emph{proteomics} & \emph{pFBA} & \emph{FVAcenter} \\
\midrule
\endfirsthead
\caption* {\textbf{Table \ref{foo} Continued}}\\\toprule
\rowcolor{white} \multirow{1}{*}{ID} & \multirow{1}{*}{Formula}  & \multicolumn{3}{c}{Fold change} \\
\rowcolor{white} & & \emph{proteomics} & \emph{pFBA} & \emph{FVAcenter} \\
\midrule
\endhead % all the lines above this will be repeated on every page
%\hline
%\multicolumn{5}{r@{}}{continued \ldots}\\
\endfoot
\hline
\endlastfoot
%
%\tablehead{%
%\toprule
%\rowcolor{white} \multirow{1}{*}{ID} & \multirow{1}{*}{Formula}  & \multicolumn{3}{c}{Fold change} \\
%%\rowcolor{white} & & \emph{prote-\\omics} & \emph{pFBA} & \emph{FVA\\center} \\
%\rowcolor{white} & & \emph{proteomics} & \emph{pFBA} & \emph{FVAcenter} \\
%\midrule}
%\tabletail{\hline}
%\rowcolors{2}{gray!25}{white}
%\begin{supertabular}{lp{0.5\textwidth}ccc}
MDH	&	mal\_\_L\_c + nad\_c $\leftrightarrow$ h\_c + nadh\_c + oaa\_c	&	1.4	&	0.2	&	0.1	\\
PEPCK\_re	&	co2\_c + gdp\_c + pep\_c $\rightarrow$ gtp\_c + oaa\_c	&	0.9	&	0.1	&	0.1	\\
VOR2b	&	3mob\_c + coa\_c + 2.0 fdxo\_42\_c $\rightarrow$ co2\_c + 2.0 fdxr\_42\_c + h\_c + ibcoa\_c	&	0.9	&	12.3	&	0.8	\\
PFL	&	coa\_c + pyr\_c $\rightarrow$ accoa\_c + for\_c	&	0.5	&	0.2	&	0.8	\\
FRNDPR2r	&	2.0 fdxr\_42\_c + h\_c + nadh\_c + 2.0 nadp\_c $\leftrightarrow$ 2.0 fdxo\_42\_c + nad\_c + 2.0 nadph\_c	&	0.3	&	0.2	&	2.3	\\
ALCD2x	&	acald\_c + h\_c + nadh\_c $\rightarrow$ etoh\_c + nad\_c	&	0.1	&	0.8	&	0.6	\\
IBUTCOARx	&	h\_c + ibcoa\_c + nadh\_c $\rightarrow$ 2mppal\_c + coa\_c + nad\_c	&	0.1	&	12.3	&	0.8	\\
ACALD	&	accoa\_c + h\_c + nadh\_c $\rightarrow$ acald\_c + coa\_c + nad\_c	&	0.1	&	0.8	&	1.1	\\
ALCD23xi	&	2mppal\_c + h\_c + nadh\_c $\rightarrow$ ibutoh\_c + nad\_c	&	0.1	&	12.3	&	0.8	\\
ME2	&	mal\_\_L\_c + nadp\_c $\rightarrow$ co2\_c + nadph\_c + pyr\_c	&	0.1	&	0.3	&	0.1	\\
PSCVT	&	pep\_c + skm5p\_c $\rightarrow$ 3psme\_c + pi\_c	&	-0.1	&	-0.3	&	0.7	\\
PTAr	&	accoa\_c + pi\_c $\leftrightarrow$ actp\_c + coa\_c	&	-0.1	&	-2.7	&	-0.1	\\
    HSOR	&	3.0 h\_c + 3.0 nadph\_c + so3\_c $\rightarrow$ 3.0 h2o\_c + h2s\_c + 3.0 nadp\_c	&	-0.1	&	-0.3	&	-1.2	\\
TRDR	&	h\_c + nadph\_c + trdox\_c $\rightarrow$ nadp\_c + trdrd\_c	&	-0.1	&	-0.3	&	0.8	\\
IGPS	&	2cpr5p\_c + h\_c $\rightarrow$ 3ig3p\_c + co2\_c + h2o\_c	&	-0.1	&	-0.3	&	0.2	\\
GLUDy	&	glu\_\_L\_c + h2o\_c + nadp\_c $\leftrightarrow$ akg\_c + h\_c + nadph\_c + nh4\_c	&	-0.1	&	-0.5	&	-0.2	\\
ECH	&	2.0 fdxr\_42\_c + 3.0 h\_c $\leftrightarrow$ 2.0 fdxo\_42\_c + h2\_c + h\_e	&	-0.1	&	-15.6	&	-13.5	\\
ALLAS	&	24.0 ala\_\_D\_c + 24.0 atp\_c + 24.0 cdpglyc\_c + dg12dg\_c + 24.0 h2o\_c $\rightarrow$ ala\_lta\_c + 24.0 amp\_c + 24.0 cmp\_c + 48.0 h\_c + 24.0 ppi\_c	&	-0.1	&	-0.2	&	0.0	\\
HSDxi	&	aspsa\_c + h\_c + nadh\_c $\rightarrow$ hom\_\_L\_c + nad\_c	&	-0.1	&	-0.3	&	0.8	\\
PRMICI	&	prfp\_c $\rightarrow$ prlp\_c	&	-0.1	&	-0.3	&	-0.1	\\
OCBT	&	cbp\_c + orn\_\_L\_c $\leftrightarrow$ citr\_\_L\_c + h\_c + pi\_c	&	-0.2	&	-0.3	&	-0.3	\\
UAGCVT	&	pep\_c + uacgam\_c $\rightarrow$ pi\_c + uaccg\_c	&	-0.2	&	-0.3	&	-0.1	\\
ANS	&	chor\_c + gln\_\_L\_c $\rightarrow$ anth\_c + glu\_\_L\_c + h\_c + pyr\_c	&	-0.2	&	-0.3	&	0.2	\\
DHAD2	&	23dhmp\_c $\rightarrow$ 3mop\_c + h2o\_c	&	-0.2	&	-0.3	&	0.6	\\
ASPO2y	&	asp\_\_L\_c + nadp\_c $\rightarrow$ h\_c + iasp\_c + nadph\_c	&	-0.3	&	-0.3	&	-0.1	\\
NADK	&	atp\_c + nad\_c $\rightarrow$ adp\_c + h\_c + nadp\_c	&	-0.3	&	-0.3	&	-0.1	\\
METS	&	5mthf\_c + hcys\_\_L\_c $\rightarrow$ met\_\_L\_c + thf\_c	&	-0.3	&	-0.3	&	0.8	\\
IMPD	&	h2o\_c + imp\_c + nad\_c $\leftrightarrow$ h\_c + nadh\_c + xmp\_c	&	-0.3	&	-0.3	&	-0.3	\\
MG2abc	&	atp\_c + h2o\_c + mg2\_e $\rightarrow$ adp\_c + h\_c + mg2\_c + pi\_c	&	-0.3	&	-0.3	&	-0.1	\\
PDHam1hi	&	h\_c + pyr\_c + thmpp\_c $\rightarrow$ 2ahethmpp\_c + co2\_c	&	-0.4	&	-0.3	&	12.4	\\
ACAS\_2ahbut	&	2ahethmpp\_c + 2obut\_c $\rightarrow$ 2ahbut\_c + thmpp\_c	&	-0.4	&	-0.3	&	0.6	\\
%BIOMASS\_CELLOBIOSE	&	0.00749443085801535 3g12dgr\_SA2\_c + 0.000534651500429844 aglu\_lta\_c + 0.226981629081658 ala\_\_L\_c + 0.00159065471272163 ala\_lta\_c + 0.0116446298806306 amp\_c + 0.152834432829489 arg\_\_L\_c + 0.19337686956005 asn\_\_L\_c + 0.203565776332897 asp\_\_L\_c + 58.55453625355648 atp\_c + 0.00431378642200049 ca2\_c + 0.000648364692685445 cdp\_c + 0.000398993657037197 clpn\_SC\_c + 0.00255887932046522 cmp\_c + 0.089081651333124 cpd11452\_c + 0.00190186976521064 cpd12556\_c + 0.0240139232409941 ctp\_c + 0.0410739907802319 cys\_\_L\_c + 0.0151967584110132 datp\_c + 0.00837186690906119 dctp\_c + 0.00966229639458412 dg12dg\_c + 0.00837186690906119 dgtp\_c + 0.0151967584110132 dttp\_c + 0.00463796876834322 fe3\_c + 0.000492757166440938 gdp\_c + 0.0907475198501415 gln\_\_L\_c + 0.267038054788532 glu\_\_L\_c + 0.000598490485555795 glu\_lta\_c + 0.238689992949084 gly\_c + 0.00315836729016361 gly\_tea\_c + 0.00364879699360517 glygly\_tea\_c + 0.00131834154179374 gmp\_c + 0.0236155245744981 gtp\_c + 58.55453625355648 h2o\_c + 0.0505986081135555 his\_\_L\_c + 0.297230937300123 ile\_\_L\_c + 0.950718761041091 k\_c + 0.30404751643422 leu\_\_L\_c + 0.00128475957565977 lta\_c + 0.283396269735403 lys\_\_L\_c + 0.088041844113358 met\_\_L\_c + 0.136964880114158 mg2\_c + 0.00272678915113504 min\_tea\_c + 0.0403931203543032 nad\_c + 0.023341128936676 nadp\_c + 0.000523014185432926 nadph\_c + 0.0154610042101914 pg\_SC\_c + 0.149324601126697 phe\_\_L\_c + 0.123899561568787 pro\_\_L\_c + 0.0491959179126864 psetha\_BS\_c + 0.214473177933542 ser\_\_L\_c + 0.00580535770989122 tg12dg\_c + 0.181493989716975 thr\_\_L\_c + 0.0331727351396739 trp\_\_L\_c + 0.149596153059727 tyr\_\_L\_c + 0.0354242930957506 utp\_c + 0.251074983559902 val\_\_L\_c $\rightarrow$ 58.55453625355648 adp\_c + 58.55453625355648 h\_c + 58.55453625355648 pi\_c + 0.161830917308754 ppi\_c	&	-0.4	&	-0.3	&	-0.1	\\
GF6PTA	&	f6p\_B\_c + gln\_\_L\_c $\rightarrow$ gam6p\_c + glu\_\_L\_c	&	-0.4	&	-0.3	&	0.0	\\
CHRS	&	3psme\_c $\rightarrow$ chor\_c + pi\_c	&	-0.4	&	-0.3	&	0.7	\\
SERH	&	3ig3p\_c + ser\_\_L\_c $\rightarrow$ g3p\_c + h2o\_c + trp\_\_L\_c	&	-0.4	&	-0.3	&	0.2	\\
ALATA\_L	&	akg\_c + ala\_\_L\_c $\leftrightarrow$ glu\_\_L\_c + pyr\_c	&	-0.5	&	-0.3	&	-12.9	\\
NNDPR	&	h\_c + prpp\_c + quln\_c $\rightarrow$ co2\_c + nicrnt\_c + ppi\_c	&	-0.5	&	-0.3	&	-0.1	\\
TMDS	&	dump\_c + mlthf\_c $\rightarrow$ dhf\_c + dtmp\_c	&	-0.5	&	-0.3	&	-0.1	\\
PHETA1	&	akg\_c + phe\_\_L\_c $\leftrightarrow$ glu\_\_L\_c + phpyr\_c	&	-0.6	&	-0.3	&	0.8	\\
TYRTA	&	akg\_c + tyr\_\_L\_c $\leftrightarrow$ 34hpp\_c + glu\_\_L\_c	&	-0.6	&	-0.3	&	0.6	\\
GMPS	&	atp\_c + nh4\_c + xmp\_c $\rightarrow$ amp\_c + gmp\_c + 3.0 h\_c + ppi\_c	&	-0.6	&	-0.3	&	-0.3	\\
SKK	&	atp\_c + skm\_c $\rightarrow$ adp\_c + h\_c + skm5p\_c	&	-0.6	&	-0.3	&	0.7	\\
ACOTA	&	acorn\_c + akg\_c $\leftrightarrow$ acg5sa\_c + glu\_\_L\_c	&	-0.6	&	-0.3	&	-0.3	\\
PPDK	&	amp\_c + 2.0 h\_c + pep\_c + ppi\_c $\rightarrow$ atp\_c + pi\_c + pyr\_c	&	-0.7	&	-10.0	&	0.1	\\
GLUPRT	&	gln\_\_L\_c + h2o\_c + prpp\_c $\rightarrow$ glu\_\_L\_c + h\_c + ppi\_c + pram\_c	&	-0.7	&	-0.3	&	-0.3	\\
KARI	&	2ahbut\_c $\leftrightarrow$ cpd10162\_c	&	-0.8	&	-0.3	&	0.6	\\
KARI\_23dhmp	&	23dhmp\_c + nadp\_c $\leftrightarrow$ cpd10162\_c + h\_c + nadph\_c	&	-0.8	&	-0.3	&	0.6	\\
ARGSL	&	argsuc\_c $\leftrightarrow$ arg\_\_L\_c + fum\_c	&	-0.8	&	-0.3	&	-0.3	\\
LEUTA	&	4mop\_c + glu\_\_L\_c $\rightarrow$ akg\_c + leu\_\_L\_c	&	-0.8	&	-0.3	&	0.4	\\
ILETA	&	akg\_c + ile\_\_L\_c $\leftrightarrow$ 3mop\_c + glu\_\_L\_c	&	-0.8	&	-0.3	&	0.6	\\
VALTA	&	akg\_c + val\_\_L\_c $\leftrightarrow$ 3mob\_c + glu\_\_L\_c	&	-0.8	&	-1.4	&	-1.5	\\
ARGSS	&	asp\_\_L\_c + atp\_c + citr\_\_L\_c $\leftrightarrow$ amp\_c + argsuc\_c + 2.0 h\_c + ppi\_c	&	-0.8	&	-0.3	&	-0.3	\\
SHSL2	&	h2s\_c + suchms\_c $\rightarrow$ hcys\_\_L\_c + succ\_c	&	-0.9	&	-6.0	&	0.8	\\
AHSL	&	achms\_c + cys\_\_L\_c $\leftrightarrow$ ac\_c + cyst\_L\_c + h\_c	&	-0.9	&	-10.6	&	-0.3	\\
SHSL1	&	cyst\_L\_c + h\_c + succ\_c $\leftrightarrow$ cys\_\_L\_c + suchms\_c	&	-0.9	&	-10.6	&	0.4	\\
ACKr	&	actp\_c + adp\_c $\rightarrow$ ac\_c + atp\_c	&	-0.9	&	-2.7	&	-0.1	\\
QULNS	&	dhap\_c + iasp\_c $\rightarrow$ 2.0 h2o\_c + h\_c + pi\_c + quln\_c	&	-0.9	&	-0.3	&	-0.1	\\
NADS2	&	atp\_c + dnad\_c + gln\_\_L\_c + h2o\_c $\rightarrow$ amp\_c + glu\_\_L\_c + 2.0 h\_c + nad\_c + ppi\_c	&	-0.9	&	-0.3	&	-0.1	\\
FE3abc	&	atp\_c + fe3\_e + h2o\_c $\rightarrow$ adp\_c + fe3\_c + h\_c + pi\_c	&	-1.0	&	-0.3	&	-0.1	\\
ASPTA	&	akg\_c + asp\_\_L\_c $\leftrightarrow$ glu\_\_L\_c + oaa\_c	&	-1.0	&	-0.3	&	0.2	\\
CTPS1	&	atp\_c + nh4\_c + utp\_c $\rightarrow$ adp\_c + ctp\_c + 2.0 h\_c + pi\_c	&	-1.2	&	-0.3	&	0.0	\\
ACGK	&	acglu\_c + atp\_c $\rightarrow$ acg5p\_c + adp\_c	&	-1.2	&	-0.3	&	-0.3	\\
IGPDH	&	eig3p\_c $\rightarrow$ h2o\_c + imacp\_c	&	-1.2	&	-0.3	&	-0.1	\\
AGPR	&	acg5sa\_c + nadp\_c + pi\_c $\leftrightarrow$ acg5p\_c + h\_c + nadph\_c	&	-1.3	&	-0.3	&	-0.3	\\
PHEt2r	&	h\_e + phe\_\_L\_e $\leftrightarrow$ h\_c + phe\_\_L\_c	&	-1.5	&	-0.3	&	0.8	\\
UAG4Ei	&	uacgam\_c $\rightarrow$ udpacgal\_c	&	-1.5	&	-0.3	&	-0.1	\\
CYSS	&	acser\_c + h2s\_c $\rightarrow$ ac\_c + cys\_\_L\_c	&	-1.8	&	-0.3	&	-0.5	\\
BIF	&	2.0 fdxr\_42\_c + 3.0 h\_c + nadh\_c $\leftrightarrow$ 2.0 fdxo\_42\_c + 2.0 h2\_c + nad\_c	&	-1.8	&	-13.8	&	-12.5	\\
UMPK	&	atp\_c + h\_c + ump\_c $\rightarrow$ adp\_c + udp\_c	&	-2.1	&	-0.3	&	0.0	\\
SULabc	&	atp\_c + h2o\_c + so4\_e $\rightarrow$ adp\_c + h\_c + pi\_c + so4\_c	&	-4.6	&	-0.3	&	0.8	\\
\end{longtable}
\normalsize




%\pagebreak
%\section{Flux simulations to further elucidate possible fate of \textit{nadph}}

\begin{table}[!ht]
\caption[Simulated fluxes for \textit{$\Delta$hydG-$\Delta$ech}]{pFBA simulated fluxes from \textit{$\Delta$hydG-$\Delta$ech} case study. Only includes reactions involving NADPH or exchange reactions that have different flux between wild-type and mutant. The biomass reaction is not included due to size. Additionally, we also excluded reactions with the same fold change magnitude as the biomass reaction ($|\textrm{FC}|=0.26$), likely because they are fully correlated.}

%\tablehead{%
%\toprule
%\rowcolor{white} \multirow{1}{*}{ID} & \multirow{1}{*}{Formula}  & \multicolumn{3}{c}{Fluxes (mmol/gCDW/hr)} \\
%\rowcolor{white} 		     &             		 & \emph{W.T.} & \emph{Mut.} 		     & FC\\
%\midrule}
%\tabletail{\hline}
\rowcolors{2}{gray!25}{white}
%\begin{supertabular}{lp{0.5\textwidth}ccc}
\begin{tabular}{lp{0.5\textwidth}ccc}
\toprule
\rowcolor{white} \multirow{1}{*}{ID} & \multirow{1}{*}{Formula}  & \multicolumn{3}{c}{Fluxes (mmol/gCDW/hr)} \\
\rowcolor{white} 		     &             		 & \emph{W.T.} & \emph{Mut.} 		     & FC\\
\midrule
EX\_ibutoh\_e	&	ibutoh\_e $\rightarrow$ 	&	0.0	&	0.49	&	12.26	\\
KARA1	&	alac\_\_S\_c + h\_c + nadph\_c $\rightarrow$ 23dhmb\_c + nadp\_c	&	0.22	&	0.59	&	1.42	\\
EX\_etoh\_e	&	etoh\_e $\rightarrow$ 	&	1.09	&	1.88	&	0.78	\\
EX\_h2o\_e	&	h2o\_e $\leftrightarrow$ 	&	-0.38	&	0.5	&	0.42	\\
EX\_co2\_e	&	co2\_e $\rightarrow$ 	&	2.07	&	2.77	&	0.42	\\
ME2	&	mal\_\_L\_c + nadp\_c $\rightarrow$ co2\_c + nadph\_c + pyr\_c	&	2.9	&	3.51	&	0.28	\\
EX\_for\_e	&	for\_e $\rightarrow$ 	&	0.48	&	0.56	&	0.23	\\
FRNDPR2r	&	2.0 fdxr\_42\_c + h\_c + nadh\_c + 2.0 nadp\_c $\leftrightarrow$ 2.0 fdxo\_42\_c + nad\_c + 2.0 nadph\_c	&	-1.02	&	-1.17	&	0.2	\\
%ASPO2y	&	asp\_\_L\_c + nadp\_c $\rightarrow$ h\_c + iasp\_c + nadph\_c	&	0.0	&	0.0	&	-0.26	\\
%EX\_phe\_\_L\_e	&	phe\_\_L\_e $\rightarrow$ 	&	0.02	&	0.02	&	-0.26	\\
%MTHFD	&	mlthf\_c + nadp\_c $\leftrightarrow$ methf\_c + nadph\_c	&	0.02	&	0.02	&	-0.26	\\
%DHFOR2	&	dhf\_c + nadp\_c $\rightarrow$ fol\_c + h\_c + nadph\_c	&	0.0	&	0.0	&	-0.26	\\
%BC13FAS	&	3mbACP\_c + 11.0 h\_c + 4.0 malACP\_c + 8.0 nadph\_c $\leftrightarrow$ 5.0 ACP\_c + branch\_c13\_fa\_c + 4.0 co2\_c + 3.0 h2o\_c + 8.0 nadp\_c	&	0.0	&	0.0	&	-0.26	\\
%BC17FAS	&	3mbACP\_c + 17.0 h\_c + 6.0 malACP\_c + 12.0 nadph\_c $\leftrightarrow$ 7.0 ACP\_c + branch\_c17\_fa\_c + 6.0 co2\_c + 5.0 h2o\_c + 12.0 nadp\_c	&	0.0	&	0.0	&	-0.26	\\
%3OAR40	&	3hbutACP\_c + nadp\_c $\leftrightarrow$ actACP\_c + h\_c + nadph\_c	&	-0.01	&	-0.01	&	-0.26	\\
%3OAR100	&	3hdecACP\_c + nadp\_c $\leftrightarrow$ 3odecACP\_c + h\_c + nadph\_c	&	-0.01	&	-0.01	&	-0.26	\\
%3OAR80	&	3hoctACP\_c + nadp\_c $\leftrightarrow$ 3ooctACP\_c + h\_c + nadph\_c	&	-0.01	&	-0.01	&	-0.26	\\
%3OAR160	&	3hpalmACP\_c + nadp\_c $\leftrightarrow$ 3opalmACP\_c + h\_c + nadph\_c	&	-0.01	&	-0.01	&	-0.26	\\
%3OAR140	&	3hmrsACP\_c + nadp\_c $\leftrightarrow$ 3omrsACP\_c + h\_c + nadph\_c	&	-0.01	&	-0.01	&	-0.26	\\
%EX\_so4\_e	&	so4\_e $\leftrightarrow$ 	&	-0.01	&	-0.01	&	-0.26	\\
%EX\_k\_e	&	k\_e $\leftrightarrow$ 	&	-0.07	&	-0.06	&	-0.26	\\
%EX\_mg2\_e	&	mg2\_e $\leftrightarrow$ 	&	-0.01	&	-0.01	&	-0.26	\\
%DHDRPy	&	nadp\_c + thdp\_c $\leftrightarrow$ 23dhdp\_c + h\_c + nadph\_c	&	-0.03	&	-0.02	&	-0.26	\\
%KARI\_23dhmp	&	23dhmp\_c + nadp\_c $\leftrightarrow$ cpd10162\_c + h\_c + nadph\_c	&	-0.02	&	-0.02	&	-0.26	\\
%ASAD	&	aspsa\_c + nadp\_c + pi\_c $\leftrightarrow$ 4pasp\_c + h\_c + nadph\_c	&	-0.05	&	-0.04	&	-0.26	\\
%3OAR60	&	3hhexACP\_c + nadp\_c $\leftrightarrow$ 3ohexACP\_c + h\_c + nadph\_c	&	-0.01	&	-0.01	&	-0.26	\\
%HSOR	&	3.0 h\_c + 3.0 nadph\_c + so3\_c $\rightarrow$ 3.0 h2o\_c + h2s\_c + 3.0 nadp\_c	&	0.01	&	0.01	&	-0.26	\\
%AGPR	&	acg5sa\_c + nadp\_c + pi\_c $\leftrightarrow$ acg5p\_c + h\_c + nadph\_c	&	-0.01	&	-0.01	&	-0.26	\\
%G5SD	&	glu5p\_c + h\_c + nadph\_c $\rightarrow$ glu5sa\_c + nadp\_c + pi\_c	&	0.01	&	0.01	&	-0.26	\\
%3OAR120	&	3hddecACP\_c + nadp\_c $\leftrightarrow$ 3oddecACP\_c + h\_c + nadph\_c	&	-0.01	&	-0.01	&	-0.26	\\
%TRDR	&	h\_c + nadph\_c + trdox\_c $\rightarrow$ nadp\_c + trdrd\_c	&	0.01	&	0.01	&	-0.26	\\
%BIOMASS\_CELLOBIOSE	&	0.00749443085801535 3g12dgr\_SA2\_c + 0.000534651500429844 aglu\_lta\_c + 0.226981629081658 ala\_\_L\_c + 0.00159065471272163 ala\_lta\_c + 0.0116446298806306 amp\_c + 0.152834432829489 arg\_\_L\_c + 0.19337686956005 asn\_\_L\_c + 0.203565776332897 asp\_\_L\_c + 58.55453625355648 atp\_c + 0.00431378642200049 ca2\_c + 0.000648364692685445 cdp\_c + 0.000398993657037197 clpn\_SC\_c + 0.00255887932046522 cmp\_c + 0.089081651333124 cpd11452\_c + 0.00190186976521064 cpd12556\_c + 0.0240139232409941 ctp\_c + 0.0410739907802319 cys\_\_L\_c + 0.0151967584110132 datp\_c + 0.00837186690906119 dctp\_c + 0.00966229639458412 dg12dg\_c + 0.00837186690906119 dgtp\_c + 0.0151967584110132 dttp\_c + 0.00463796876834322 fe3\_c + 0.000492757166440938 gdp\_c + 0.0907475198501415 gln\_\_L\_c + 0.267038054788532 glu\_\_L\_c + 0.000598490485555795 glu\_lta\_c + 0.238689992949084 gly\_c + 0.00315836729016361 gly\_tea\_c + 0.00364879699360517 glygly\_tea\_c + 0.00131834154179374 gmp\_c + 0.0236155245744981 gtp\_c + 58.55453625355648 h2o\_c + 0.0505986081135555 his\_\_L\_c + 0.297230937300123 ile\_\_L\_c + 0.950718761041091 k\_c + 0.30404751643422 leu\_\_L\_c + 0.00128475957565977 lta\_c + 0.283396269735403 lys\_\_L\_c + 0.088041844113358 met\_\_L\_c + 0.136964880114158 mg2\_c + 0.00272678915113504 min\_tea\_c + 0.0403931203543032 nad\_c + 0.023341128936676 nadp\_c + 0.000523014185432926 nadph\_c + 0.0154610042101914 pg\_SC\_c + 0.149324601126697 phe\_\_L\_c + 0.123899561568787 pro\_\_L\_c + 0.0491959179126864 psetha\_BS\_c + 0.214473177933542 ser\_\_L\_c + 0.00580535770989122 tg12dg\_c + 0.181493989716975 thr\_\_L\_c + 0.0331727351396739 trp\_\_L\_c + 0.149596153059727 tyr\_\_L\_c + 0.0354242930957506 utp\_c + 0.251074983559902 val\_\_L\_c $\rightarrow$ 58.55453625355648 adp\_c + 58.55453625355648 h\_c + 58.55453625355648 pi\_c + 0.161830917308754 ppi\_c	&	0.07	&	0.06	&	-0.26	\\
%EX\_fe3\_e	&	fe3\_e $\leftrightarrow$ 	&	-0.0	&	-0.0	&	-0.26	\\
%3OAR18B	&	3osteACP\_c + h\_c + nadph\_c $\leftrightarrow$ 3hoctaACP\_c + nadp\_c	&	0.0	&	0.0	&	-0.26	\\
%EX\_ca2\_e	&	ca2\_e $\leftrightarrow$ 	&	-0.0	&	-0.0	&	-0.26	\\
%EX\_pi\_e	&	pi\_e $\leftrightarrow$ 	&	-0.09	&	-0.08	&	-0.26	\\
EX\_nh4\_e	&	nh4\_e $\leftrightarrow$ 	&	-0.73	&	-0.53	&	-0.48	\\
GLUDy	&	glu\_\_L\_c + h2o\_c + nadp\_c $\leftrightarrow$ akg\_c + h\_c + nadph\_c + nh4\_c	&	-0.61	&	-0.42	&	-0.53	\\
EX\_h\_e	&	h\_e $\leftrightarrow$ 	&	2.93	&	1.19	&	-1.3	\\
EX\_val\_\_L\_e	&	val\_\_L\_e $\rightarrow$ 	&	0.18	&	0.06	&	-1.54	\\
ICDHyr	&	icit\_c + nadp\_c $\rightarrow$ akg\_c + co2\_c + nadph\_c	&	0.21	&	0.04	&	-2.27	\\
EX\_ac\_e	&	ac\_e $\rightarrow$ 	&	0.93	&	0.17	&	-2.46	\\
EX\_lac\_\_L\_e	&	lac\_\_L\_e $\rightarrow$ 	&	0.05	&	0.01	&	-2.49	\\
EX\_succ\_e	&	succ\_e $\rightarrow$ 	&	0.41	&	0.0	&	-12.01	\\
EX\_h2\_e	&	h2\_e $\rightarrow$ 	&	2.2	&	0.0	&	-14.43	\\
%\end{supertabular}
\end{tabular}

\end{table}

